%
\begin{isabellebody}%
\setisabellecontext{Graph}%
%
\isadelimdocument
%
\endisadelimdocument
%
\isatagdocument
%
\isamarkupsection{Graph%
}
\isamarkuptrue%
%
\endisatagdocument
{\isafolddocument}%
%
\isadelimdocument
%
\endisadelimdocument
%
\isadelimtheory
%
\endisadelimtheory
%
\isatagtheory
\isacommand{theory}\isamarkupfalse%
\ Graph\isanewline
\ \ \isakeyword{imports}\isanewline
\ \ \ \ {\isachardoublequoteopen}Adjacency{\isacharslash}{\kern0pt}Adjacency{\isachardoublequoteclose}\isanewline
\ \ \ \ {\isachardoublequoteopen}Adjacency{\isacharslash}{\kern0pt}Adjacency{\isacharunderscore}{\kern0pt}Impl{\isachardoublequoteclose}\isanewline
\ \ \ \ {\isachardoublequoteopen}Directed{\isacharunderscore}{\kern0pt}Graph{\isacharslash}{\kern0pt}Directed{\isacharunderscore}{\kern0pt}Graph{\isachardoublequoteclose}\isanewline
\ \ \ \ {\isachardoublequoteopen}Undirected{\isacharunderscore}{\kern0pt}Graph{\isacharslash}{\kern0pt}Undirected{\isacharunderscore}{\kern0pt}Graph{\isachardoublequoteclose}\isanewline
\isakeyword{begin}%
\endisatagtheory
{\isafoldtheory}%
%
\isadelimtheory
%
\endisadelimtheory
%
\begin{isamarkuptext}%
This section considers graphs from three levels of abstraction. On the high level, a graph is a set
of edges (\isa{graph} for undirected graphs, and \isa{dgraph} for directed graphs). On the
medium level, a graph is specified via the interface \isa{adjacency}. On the low level, this
interface is then implemented via red-black trees.%
\end{isamarkuptext}\isamarkuptrue%
%
\isadelimdocument
%
\endisadelimdocument
%
\isatagdocument
%
\isamarkupsubsection{High level%
}
\isamarkuptrue%
%
\endisatagdocument
{\isafolddocument}%
%
\isadelimdocument
%
\endisadelimdocument
%
\begin{isamarkuptext}%
For the high level of abstraction, we extend the archive of graph formalizations \isatt{A{\kern0pt}G{\kern0pt}F{\kern0pt}},
which formalizes both directed (\isa{dgraph}) and undirected (\isa{graph}) graphs as sets of
edges. The set of vertices of a graph is then defined as the union of all endpoints of all edges in
the graph (\isa{dVs\ {\isacharquery}{\kern0pt}dG\ {\isasymequiv}\ {\isasymUnion}\ {\isacharbraceleft}{\kern0pt}{\isacharbraceleft}{\kern0pt}v{\isadigit{1}}{\isacharcomma}{\kern0pt}\ v{\isadigit{2}}{\isacharbraceright}{\kern0pt}\ {\isacharbar}{\kern0pt}v{\isadigit{1}}\ v{\isadigit{2}}{\isachardot}{\kern0pt}\ v{\isadigit{1}}\ {\isasymrightarrow}\isactrlbsub {\isacharquery}{\kern0pt}dG\isactrlesub v{\isadigit{2}}{\isacharbraceright}{\kern0pt}} for directed graphs, and \isa{Vs\ {\isacharquery}{\kern0pt}E\ {\isasymequiv}\ {\isasymUnion}\ {\isacharquery}{\kern0pt}E} for undirected graphs). Let us
first look at directed graphs.%
\end{isamarkuptext}\isamarkuptrue%
%
\isadelimtheory
%
\endisadelimtheory
%
\isatagtheory
\isacommand{end}\isamarkupfalse%
%
\endisatagtheory
{\isafoldtheory}%
%
\isadelimtheory
%
\endisadelimtheory
%
\end{isabellebody}%
\endinput
%:%file=~/Documents/1_projects/2022_idp/thys/Graph/Graph.thy%:%
%:%11=1%:%
%:%27=3%:%
%:%28=3%:%
%:%29=4%:%
%:%30=5%:%
%:%31=6%:%
%:%32=7%:%
%:%33=8%:%
%:%34=9%:%
%:%43=12%:%
%:%44=13%:%
%:%45=14%:%
%:%46=15%:%
%:%55=18%:%
%:%67=21%:%
%:%68=22%:%
%:%69=23%:%
%:%70=24%:%
%:%71=25%:%
%:%79=28%:%